\documentclass[a4paper, 12pt]{article}
%%% Работа с русским языком
\usepackage{cmap}					% поиск в PDF
\usepackage{mathtext} 				% русские буквы в формулах
\usepackage[T2A]{fontenc}			% кодировка
\usepackage[utf8]{inputenc}			% кодировка исходного текста
\usepackage[russian]{babel}	% локализация и переносы

%%% Дополнительная работа с математикой
\usepackage[shortlabels]{enumitem}
\usepackage{amsmath,amsfonts,amssymb,amsthm,mathtools} % AMS
\usepackage{icomma} % "Умная" запятая: $0,2$ --- число, $0, 2$ --- перечисление

%% Номера формул
%\mathtoolsset{showonlyrefs=true} % Показывать номера только у тех формул, на которые есть \eqref{} в тексте.
%\usepackage{leqno} % Немуреация формул слева

%% Шрифты
\usepackage{euscript}	 % Шрифт Евклид
\usepackage{mathrsfs} % Красивый матшрифт
\usepackage{dsfont}
\usepackage{upgreek}

%%% Свои команды
\DeclareMathOperator{\sgn}{\mathop{sgn}}

%% Поля
\usepackage[left=2cm,right=2cm,top=2cm,bottom=2cm,bindingoffset=0cm]{geometry}

%% Русские списки
\usepackage{enumitem}
\makeatletter
\AddEnumerateCounter{\asbuk}{\russian@alph}{щ}
\makeatother

%%% Работа с картинками
\usepackage{graphicx}  % Для вставки рисунков
\graphicspath{{images/}{images2/}}  % папки с картинками
\setlength\fboxsep{3pt} % Отступ рамки \fbox{} от рисунка
\setlength\fboxrule{1pt} % Толщина линий рамки \fbox{}
\usepackage{wrapfig} % Обтекание рисунков и таблиц текстом

%%% Работа с таблицами
\usepackage{array,tabularx,tabulary,booktabs} % Дополнительная работа с таблицами
\usepackage{longtable}  % Длинные таблицы
\usepackage{multirow} % Слияние строк в таблице

%% Красная строка
\setlength{\parindent}{2em}

%% Интервалы
\linespread{1}
\usepackage{multirow}

%% TikZ
\usepackage{tikz}
\usetikzlibrary{graphs,graphs.standard}

%% Верхний колонтитул
% \usepackage{fancyhdr}
% \pagestyle{fancy}

%% Перенос знаков в формулах (по Львовскому)
\newcommand*{\hm}[1]{#1\nobreak\discretionary{}
	{\hbox{$\mathsurround=0pt #1$}}{}}

%% дополнения
\usepackage{float} %Добавляет возможность работы с командой [H] которая улучшает расположение на странице
\usepackage{gensymb} %Красивые градусы
\usepackage{caption} % Пакет для подписей к рисункам, в частности, для работы caption*

% подключаем hyperref (для ссылок внутри  pdf)
\usepackage[unicode, pdftex]{hyperref}

\numberwithin{equation}{section}
\numberwithin{equation}{subsection}

%%% Теоремы
\theoremstyle{plain}                    % Это стиль по умолчанию, его можно не переопределять.
\renewcommand\qedsymbol{$\blacksquare$} % переопределение символа завершения доказательства

\newtheorem{theorem}{Теорема} % Теорема (счетчик по секиям)
\newtheorem{proposition}{Утверждение} % Утверждение (счетчик по секиям)
\newtheorem{definition}{Определение} % Определение (счетчик по секиям)
\newtheorem{corollary}{Следствие}[theorem] % Следстиве (счетчик по теоремам)
\newtheorem*{consequence}{Следствие}
\newtheorem{problem}{Упражнение} % Задача (счетчик по секиям)
\newtheorem{problem*}{Упражнение*}
\newtheorem*{remark}{Замечание} % Примечание (можно переопределить, как Замечание)
\newtheorem{lemma}{Лемма} % Лемма (счетчик по секиям)
\newtheorem*{designation}{Обозначение} % Лемма (счетчик по секиям)

\newtheorem{example}{Пример} % Пример
\newtheorem{counterexample}{Контрпример} % Контрпример

% from package mathabx:
\DeclareFontFamily{U}{mathb}{\hyphenchar\font45}
\DeclareFontShape{U}{mathb}{m}{n}{
      <5> <6> <7> <8> <9> <10> gen * mathb
      <10.95> mathb10 <12> <14.4> <17.28> <20.74> <24.88> mathb12
      }{}
\DeclareSymbolFont{mathb}{U}{mathb}{m}{n}
\DeclareFontSubstitution{U}{mathb}{m}{n}
\DeclareMathSymbol{\downtouparrow}{3}{mathb}{"FF}

\newcommand*{\rotarrow}{\mathbin{\rotatebox[origin=c]{180}{$\curvearrowleft$}}}

\newcommand*{\cuparr}[2]{%
  \underset{#1}{%
    \textstyle
    \overset{#2}{\rotarrow}%
  }%
}

%   \newcommand{\carrowover}[2][2ex]{\stepcounter{carrowover}\tikzset{tikzmark
% prefix=\thecarrowover}\tikzmark{start}#2\tikzmark{stop}\tikz[remember
% picture, overlay]{\draw[->]([shift={(.5ex,#1)}]pic cs:start) to[bend
% left = 180] ([shift={(-.5ex,#1)}]pic cs:stop);}}

\newcommand{\defeq}{\stackrel{def}{=}} % по определению
\newcommand{\defarr}{\stackrel{def}{\Rightarrow}} % следует из определения

\makeatletter
\newcommand{\eqnum}{\refstepcounter{equation}\textup{\tagform@{\theequation}}}
\makeatother % создание метки и нумерация формулы одновременно

\newcommand{\deflimk}{\lim\limits_{k\rightarrow \infty}} % лимит при k -> бесконечности
\DeclareMathOperator{\Tr}{trace} % след матрицы
\DeclareMathOperator{\diverg}{div} % определение нормально выглядещей дивергенции
\newcommand{\bignu}{\nu} %Большое ню, если кто-то потом узнает, как его писать

\usepackage{titlesec}
\titlelabel{\thetitle.\quad}

\begin{document}
    
%=======================================================================================

\begin{titlepage}
\begin{center}
\

\large\textbf{Московский Физико-Технический Институт}\\
\large\textbf{(государственный университет)}
\vfill

{\LARGE \textsc{\textbf{Дискретные случайные процессы.\\ Лекция 8.\\}}}

\vspace{10em}


\begin{flushright}
    \normalsize{Выполнил: Дурнов Алексей Николаевич \\ студент Б01-009\\}
\end{flushright}
\vfill

Долгопрудный, 2023
\end{center}
\end{titlepage}

%=======================================================================================

\newpage


\begin{remark}
    Теорема о единственном продолжении неотрицательной ограниченной счётно-аддитивной меры с кольца на порождённое им $\sigma$-кольцо с сохранением $sigma$-аддитивности меры - \underline{главная теорема курса}
\end{remark}

\begin{designation}
Пусть $K$ - множественное кольцо $\xrightarrow[\sigma_{\text{адд}}]{\mu} [0, N]$. Тогда $\bigcup K := E := \bigcup \limits_{A \in K} A = \bigcup \{A | A \in K\} $.
\end{designation}

\begin{definition}
    Множество $M \in E$ назовём $K_{\sigma}$-покрываемым, если $\exists (A_1, A_2, \dots) \in K^{\mathbb{N}}$ такое, 
    что $M \subset \bigcup \limits_{k = 1}^{\infty} A_k \equiv \cuparr{n = 1}{\infty} ( \underbrace{\bigcup \limits_{k = 1}^{n} A_k}_{\in K} ) \equiv \bigsqcup \limits_{m = 1}^{\infty} ( \underbrace{A_m \setminus  \bigcup \left\{ A_k \big|_: k < m \right\}}_{\in K} ) $.
\end{definition}

\begin{designation}
    $\mathcal{P}_{K_{\sigma}}(E) = \left\{M |\, M \text{ является } K_{\sigma}\text{-покрываемым}\right\}$.
\end{designation}

$$\mu^*(M) = \inf \{\sum \limits_{k = 1}^{\infty} \mu(C_k) |\, \forall k \in N\, C_k \in K;\; M \subset \bigsqcup \limits_{k = 1}^{\infty} C_k \} \leqslant N$$
$$\rho^*_{\mu}: \mathcal{P}_{K_{\sigma}}(E) \times \mathcal{P}_{K_{\sigma}}(E) \rightarrow [0, N]: (A, B) \rightarrow \mu^*(A \Delta B), \, \forall M \in \mathcal{P}_{K_{\sigma}}(E)$$ 
$$\mu_*(M) = \sup \{ \mu(A) |\, A \in K, \, A \subset M  \} \leqslant N$$


Будем называть $\mu^*$ внешней мерой, а $\mu_*$ - внутренней.

\begin{definition}
    $\mu*$-измеримыми подмножества в $E \equiv \bigcup K$ назовём такие $M \in \mathcal{P}_{K_{\sigma}}(E)$, что для некоторой $(A_1, A_2, \dots, A_n, \dots) \in K^{\mathbb{N}}$ такие, что $M \subset \bigcup \limits_{k = 1}^{\infty} A_k \equiv U$, выполнено соотношение:
    $$ \mu^* (M) + \mu^*(U \setminus M) = \mu^*(U) \equiv \sup \limits_{n \in \mathbb{N}}\left\{ \mu \left( \bigcup \limits_{k = 1}^{n} A_k \right)\right\} (\text{конструкция Каратеодори})$$
\end{definition}

\begin{definition}
    $\rho_{\mu}^*$-измеримым называется такое $M \in \mathcal{P}_{K_{\sigma}}(E)$, что $\exists (A_{k,l})_{k,l = 1}^{\infty} \in K^{\mathbb{N} \times \mathbb{N}}$, что $\rho_{\mu}^*\left(\bigcap \limits_{l=1}^{\infty}\left(\bigcup \limits_{k = 1}^{\infty} A_{k,l} \right), M\right) = 0$ (Лебег).
\end{definition}

\begin{problem*}
    Система $\mu^*$-измеримых подмножеств в $E$ совпадает с системой $\rho^*_{\mu}$-измеримый подмножеств $E$, является $\sigma$-кольцом, которое называется $\sigma$-кольцом множеств, измеримых относительно естественного продолжения меры $\mu$, обозначим $\bar{K}^{\mu}$.

    $$ \mu^* \big|_{\bar{K}^{\mu}} =: \bar{\mu} - ~\text{$\sigma$-адд}$$
    $$ \bar{\mu} \big|_{K} = \mu^* \big|_{K^{\mu}} = \mu$$
\end{problem*}

\begin{remark}
    Вообще говоря, $\mu^* \big|_{\bar{K}^{\mu}} \neq \bar{\mu}$
\end{remark}

\begin{remark}
   $\rho_{\bar{\mu}}$ порождает ту же самую метрику на фактор пространстве $\bar{K}/\tilde{\rho}_{\bar{\mu}}$, что и копия аналогично получаемой метрики из пополнения полуметрического пространства $(K, \rho_{\mu})$, т.е. классу эквивалентных фундаментальных последовательностей будет соответствовать один класс эквивалентных измеримых множеств из $(\bar{K}^{\mu}, \rho_{\bar{\mu}})$.
\end{remark}

\begin{designation}
    $\mathscr{B}(\mathbb{R}^n)$ - борелевская $\sigma$-алгебра, т.е. наименьшее $\sigma$-кольцо или $\sigma$-алгебра, содержащая все декартовы произведения одномерных промежутков. Она же порождается и открытами шарами, и просто открытыми множествами в $\mathbb{R}^d$, и значит замкнутыми.
\end{designation}


\subsection*{Конструкция $d$-мерной меры Лебега.}

\begin{enumerate}[a)]
    \item  1
    \item  2
    \item 
\end{enumerate}

\begin{theorem}
    Определение $\lambda ^ d$ корректно: $\lambda ^ d$ называется полной $d$-мерной мерой.
\end{theorem}

\begin{remark}
$\mathtt{Card}(B(\mathbb{R}^d)) = \complement$ (continuum), где $\mathfrak{Card}$ - количество элементов (кардинальное число), мощность.
\end{remark}



\end{document}
