\documentclass[a4paper, 12pt]{article}
%%% Работа с русским языком
\usepackage{cmap}					% поиск в PDF
\usepackage{mathtext} 				% русские буквы в формулах
\usepackage[T2A]{fontenc}			% кодировка
\usepackage[utf8]{inputenc}			% кодировка исходного текста
\usepackage[russian]{babel}	% локализация и переносы

%%% Дополнительная работа с математикой
\usepackage[shortlabels]{enumitem}
\usepackage{amsmath,amsfonts,amssymb,amsthm,mathtools} % AMS
\usepackage{icomma} % "Умная" запятая: $0,2$ --- число, $0, 2$ --- перечисление

%% Номера формул
%\mathtoolsset{showonlyrefs=true} % Показывать номера только у тех формул, на которые есть \eqref{} в тексте.
%\usepackage{leqno} % Немуреация формул слева

%% Шрифты
\usepackage{euscript}	 % Шрифт Евклид
\usepackage{mathrsfs} % Красивый матшрифт
\usepackage{dsfont}
\usepackage{upgreek}

%%% Свои команды
\DeclareMathOperator{\sgn}{\mathop{sgn}}

%% Поля
\usepackage[left=2cm,right=2cm,top=2cm,bottom=2cm,bindingoffset=0cm]{geometry}

%% Русские списки
\usepackage{enumitem}
\makeatletter
\AddEnumerateCounter{\asbuk}{\russian@alph}{щ}
\makeatother

%%% Работа с картинками
\usepackage{graphicx}  % Для вставки рисунков
\graphicspath{{images/}{images2/}}  % папки с картинками
\setlength\fboxsep{3pt} % Отступ рамки \fbox{} от рисунка
\setlength\fboxrule{1pt} % Толщина линий рамки \fbox{}
\usepackage{wrapfig} % Обтекание рисунков и таблиц текстом

%%% Работа с таблицами
\usepackage{array,tabularx,tabulary,booktabs} % Дополнительная работа с таблицами
\usepackage{longtable}  % Длинные таблицы
\usepackage{multirow} % Слияние строк в таблице

%% Красная строка
\setlength{\parindent}{2em}

%% Интервалы
\linespread{1}
\usepackage{multirow}

%% TikZ
\usepackage{tikz}
\usetikzlibrary{graphs,graphs.standard}

%% Верхний колонтитул
% \usepackage{fancyhdr}
% \pagestyle{fancy}

%% Перенос знаков в формулах (по Львовскому)
\newcommand*{\hm}[1]{#1\nobreak\discretionary{}
	{\hbox{$\mathsurround=0pt #1$}}{}}

%% дополнения
\usepackage{float} %Добавляет возможность работы с командой [H] которая улучшает расположение на странице
\usepackage{gensymb} %Красивые градусы
\usepackage{caption} % Пакет для подписей к рисункам, в частности, для работы caption*

% подключаем hyperref (для ссылок внутри  pdf)
\usepackage[unicode, pdftex]{hyperref}

\numberwithin{equation}{section}
\numberwithin{equation}{subsection}

%%% Теоремы
\theoremstyle{plain}                    % Это стиль по умолчанию, его можно не переопределять.
\renewcommand\qedsymbol{$\blacksquare$} % переопределение символа завершения доказательства

\newtheorem{theorem}{Теорема} % Теорема (счетчик по секиям)
\newtheorem{proposition}{Утверждение} % Утверждение (счетчик по секиям)
\newtheorem{definition}{Определение} % Определение (счетчик по секиям)
\newtheorem{corollary}{Следствие}[theorem] % Следстиве (счетчик по теоремам)
\newtheorem*{consequence}{Следствие}
\newtheorem{problem}{Упражнение} % Задача (счетчик по секиям)
\newtheorem{problem*}{Упражнение*}
\newtheorem*{remark}{Замечание} % Примечание (можно переопределить, как Замечание)
\newtheorem{lemma}{Лемма} % Лемма (счетчик по секиям)
\newtheorem*{designation}{Обозначение} % Лемма (счетчик по секиям)

\newtheorem{example}{Пример} % Пример
\newtheorem{counterexample}{Контрпример} % Контрпример

% from package mathabx:
\DeclareFontFamily{U}{mathb}{\hyphenchar\font45}
\DeclareFontShape{U}{mathb}{m}{n}{
      <5> <6> <7> <8> <9> <10> gen * mathb
      <10.95> mathb10 <12> <14.4> <17.28> <20.74> <24.88> mathb12
      }{}
\DeclareSymbolFont{mathb}{U}{mathb}{m}{n}
\DeclareFontSubstitution{U}{mathb}{m}{n}
\DeclareMathSymbol{\downtouparrow}{3}{mathb}{"FF}

\newcommand*{\rotarrow}{\mathbin{\rotatebox[origin=c]{180}{$\curvearrowleft$}}}

\newcommand*{\cuparr}[2]{%
  \underset{#1}{%
    \textstyle
    \overset{#2}{\rotarrow}%
  }%
}

%   \newcommand{\carrowover}[2][2ex]{\stepcounter{carrowover}\tikzset{tikzmark
% prefix=\thecarrowover}\tikzmark{start}#2\tikzmark{stop}\tikz[remember
% picture, overlay]{\draw[->]([shift={(.5ex,#1)}]pic cs:start) to[bend
% left = 180] ([shift={(-.5ex,#1)}]pic cs:stop);}}

\newcommand{\defeq}{\stackrel{def}{=}} % по определению
\newcommand{\defarr}{\stackrel{def}{\Rightarrow}} % следует из определения

\makeatletter
\newcommand{\eqnum}{\refstepcounter{equation}\textup{\tagform@{\theequation}}}
\makeatother % создание метки и нумерация формулы одновременно

\newcommand{\deflimk}{\lim\limits_{k\rightarrow \infty}} % лимит при k -> бесконечности
\DeclareMathOperator{\Tr}{trace} % след матрицы
\DeclareMathOperator{\diverg}{div} % определение нормально выглядещей дивергенции
\newcommand{\bignu}{\nu} %Большое ню, если кто-то потом узнает, как его писать

\usepackage{titlesec}
\titlelabel{\thetitle.\quad}

\begin{document}
    
%=======================================================================================

\begin{titlepage}
\begin{center}
\

\large\textbf{Московский Физико-Технический Институт}\\
\large\textbf{(государственный университет)}
\vfill

{\LARGE \textsc{\textbf{Дискретные случайные процессы.\\ Лекция 8.\\}}}

\vspace{10em}


\begin{flushright}
    \normalsize{Выполнил: Дурнов Алексей Николаевич \\ студент Б01-009\\}
\end{flushright}
\vfill

Долгопрудный, 2023
\end{center}
\end{titlepage}

%=======================================================================================

\newpage


\begin{remark}
    Теорема о единственном продолжении неотрицательной ограниченной счётно-аддитивной меры с кольца на порождённое им $\sigma$-кольцо с сохранением $\sigma$-аддитивности меры -- \underline{главная теорема курса}.
\end{remark}

\begin{designation}
Пусть $K$ -- множественное кольцо $\xrightarrow[\sigma_{\text{адд}}]{\mu} [0, N]$. 
$$ \text{Тогда } \bigcup K := E := \bigcup \limits_{A \in K} A = \bigcup \{A\, |\, A \in K\}.$$
\end{designation}

\begin{definition}
    Множество $M \in E$ назовём $K_{\sigma}$-покрываемым, если $\exists (A_1, A_2, \dots) \in K^{\mathbb{N}}$ такая, 
    что $M \subset \bigcup \limits_{k = 1}^{\infty} A_k \equiv \cuparr{n = 1}{\infty} ( \underbrace{\bigcup \limits_{k = 1}^{n} A_k}_{\in K} ) \equiv \bigsqcup \limits_{m = 1}^{\infty} ( \underbrace{A_m \setminus  \bigcup \left\{ A_k \, \big|_: \, k < m \right\}}_{\in K} ) $.
\end{definition}

\begin{designation}
    $\mathcal{P}_{K_{\sigma}}(E) = \left\{M \,\big|\, M \text{ является } K_{\sigma}\text{-покрываемым}\right\}$.
\end{designation}

$$\mu^*(M) = \inf \left\{\sum \limits_{k = 1}^{\infty} \mu(C_k) \,\big|\, \forall k \in N\, C_k \in K;\; M \subset \bigsqcup \limits_{k = 1}^{\infty} C_k \right\} \leqslant N$$
$$\rho^*_{\mu}: \mathcal{P}_{K_{\sigma}}(E) \times \mathcal{P}_{K_{\sigma}}(E) \rightarrow [0, N]: (A, B) \rightarrow \mu^*(A \Delta B),$$ 
$$\forall M \in \mathcal{P}_{K_{\sigma}}(E) \,\, \mu_*(M) = \sup \{ \mu(A) |\, A \in K, \, A \subset M  \} \leqslant N$$


Будем называть $\mu^*$ внешней мерой, а $\mu_*$ -- внутренней.

\begin{definition}
    $\mu^*$-измеримыми подмножества в $E \equiv \bigcup K$ назовём такие $M \in \mathcal{P}_{K_{\sigma}}(E)$, что для некоторой $(A_1, A_2, \dots, A_n, \dots) \in K^{\mathbb{N}}$ такая, что $M \subset \bigcup \limits_{k = 1}^{\infty} A_k \equiv U$, выполнено соотношение:
    $$ \mu^* (M) + \mu^*(U \setminus M) = \mu^*(U) \equiv \sup \limits_{n \in \mathbb{N}}\left\{ \mu \left( \bigcup \limits_{k = 1}^{n} A_k \right)\right\} (\text{конструкция Каратеодори}).$$
\end{definition}

\begin{definition}
    $\rho_{\mu}^*$-измеримым называется такое $M \in \mathcal{P}_{K_{\sigma}}(E)$, что $\exists (A_{k,l})_{k,l = 1}^{\infty} \in K^{\mathbb{N} \times \mathbb{N}}$, что 
    $$\rho_{\mu}^*\left(\bigcap \limits_{l=1}^{\infty}\left(\bigcup \limits_{k = 1}^{\infty} A_{k,l} \right), M\right) = 0 \text{ (Лебег).}$$
\end{definition}

\begin{problem}[*]
    Система $\mu^*$-измеримых подмножеств в $E$ совпадает с системой $\rho^*_{\mu}$-измеримых подмножеств в $E$, является $\sigma$-кольцом, которое называется $\sigma$-кольцом множеств, измеримых относительно естественного продолжения меры $\mu$, обозначим $\bar{K}^{\mu}$.

    $$ \mu^* \big|_{\bar{K}^{\mu}} =: \bar{\mu} - ~\text{$\sigma$-аддитивна}$$
    $$ \bar{\mu} \big|_{K} = \mu^* \big|_{K^{\mu}} = \mu.$$
\end{problem}

\begin{remark}
    Вообще говоря, $\mu^* \big|_{\bar{K}^{\mu}} \neq \bar{\mu}$.
\end{remark}

\begin{remark}
   $\rho_{\bar{\mu}}$ порождает ту же самую метрику на фактор пространстве $\bar{K}/\tilde{\rho}_{\bar{\mu}}$, что и копия аналогично получаемой метрики из пополнения полуметрического пространства $(K, \rho_{\mu})$, т.е. классу эквивалентных фундаментальных последовательностей будет соответствовать один класс эквивалентных измеримых множеств из $(\bar{K}^{\mu}, \rho_{\bar{\mu}})$.
\end{remark}

\begin{designation}
    $\mathscr{B}(\mathbb{R}^n)$ -- борелевская $\sigma$-алгебра, т.е. наименьшее $\sigma$-кольцо или $\sigma$-алгебра, содержащая все декартовы произведения одномерных промежутков. Она же порождается и открытыми шарами, и просто открытыми множествами в $\mathbb{R}^d$, и значит замкнутыми.
\end{designation}


\subsection*{Конструкция $d$-мерной меры Лебега.}

\begin{enumerate}[a)]
    \item  
    \begin{definition}
        Рассматриваем те ограниченные множества $M \subset [-N, N]^d \subset \mathbb{R}^d$, что $\mathds{1}_{M \subset [-N, N]^d} \in R \left([-N, N]^d \right)$, такие $M$ называются измеримыми по Жордану, 
        $$ \int \limits_{-N}^{N} \dots \int \limits_{-N}^{N} \mathds{1}_{M \subset [-N, N]^d} (x_1, ..., x_d) dx_1 \dots dx_d = |M|_d - \text{ $d$-мерная мера Жордана.}$$
    \end{definition}
    \item Кольцо всех измеримых по Жордану подмножеств в фиксированном кубе не является $\sigma$-кольцом, но мера Жордана ($d$-мерная) на этом кольце счётно-аддитивная, и мы можем по теореме продолжить:
    
    Пусть $J([-N, N]^d)$ -- кольцо измеримых по Жордану подмножеств в $[-N, N]^d$, $\lambda_{J([-N, N]^d)} \equiv \lambda_{J, N}^d: J([-N, N]^d) \rightarrow [0, +\infty): M \rightarrow |M|_d$, тогда $\overline{\lambda_{J, N}^d} \equiv \lambda^d_N : \overline{J([-N, N]^d)}^{\lambda_{J, N}^{d}} \equiv M_{\lambda, N}^d \rightarrow [0, (2N)^d]$, где $\overline{J([-N, N]^d)}^{\lambda_{J, N}^{d}}$ -- измеримые по Лебегу подмножества куба.

    \begin{remark}
        $0 < N_1 < N_2 \Rightarrow \forall m \in M_{\lambda, N_1}^d \, \lambda_{N_1}^d(m) = \lambda_{N_2}^d(m)$, т.е $\lambda_{N_1}^d = \lambda_{N_2} \big|_{M_{\lambda, N_1}^d}$.
    \end{remark}

    \begin{problem}
        $$\sigma_{\varkappa} \left(\bigcup \limits_{N > 0} M_{\lambda, N}^d \right) = \sigma_{\varkappa} \left(\bigcup \limits_{N \in \mathbb{N}} M_{\lambda, N}^d \right) = \left\{ \bigcup \limits_{k = 1}^{\infty} A_k \,\big|\, A_k \in M_{\lambda, N}^d \right\} \equiv M_{\lambda}^d \overset{*}{\Rightarrow} \equiv \mathscr{B}(\mathbb{R}^d),$$
        где $M_{\lambda}^d$ - система всех измеримых относительнго (полной) меры Лебега в $\mathbb{R}^d$ подмножеств.
    \end{problem}

    \begin{definition}
        $\lambda^d: \, M_{\lambda}^d \rightarrow [0, +\infty]$, такая что $\lambda^d(\cuparr{k = 1}{\infty} A_K) = \lim \limits_{k \rightarrow \infty} \lambda_{k}^d(A_k)$ при $A_k \in M_{\lambda, k}^d$.
        $\lambda ^ d$ называется полной $d$-мерной мерой Лебега.
    \end{definition}

    \begin{theorem}
        Определение $\lambda ^ d$ корректно.
    \end{theorem}
    
    \begin{remark}
        $Card(\mathcal{B}(\mathbb{R}^d)) = \complement$ (continuum), где $Card$ -- количество элементов (кардинальное число), мощность.
    \end{remark}

    \begin{example}
        Канторово стандартное множество измеримо по Жордану, и все подмножества тоже, а этих подмножеств $\supset \complement$ 
        % Для замкнутого множества внешняя мера Жордана равна мере Лебега. Для открытого множества внутренняя мера Жордана равна мере Лебега.
    \end{example}

    \begin{definition}
        Пусть $S$ -- система множеств, $\mu:\, S \rightarrow [0, +\infty]$, $\mu$ называется полной, если $\forall A \in S(\mu (A) = 0 \Rightarrow \forall B \subset A: \, B \in S \,\&\, \mu (B) = 0)$. В этом случае и саму систему $S$ называют полной относительно меры $\mu$.
    \end{definition}

    \begin{consequence}
        $\lambda^d$ - полна, $M_{\lambda}^d$ -- полна относительно $\lambda^d$.
    \end{consequence}
    
    \begin{definition}
        $\sigma$-аддитивная неотрицательная мера на $\sigma$-алгребре $\text{\textscripta}$ с единицей $E$ называется $\sigma$-конечной, если $\exists (A_1, A_2, \dots) \in \text{\textscripta}^{\mathbb{N}}$ такие, что
        \begin{enumerate}[1)]
            \item $\cuparr{n = 1}{\infty} A_n = E$
            \item $\forall n \in N\, \mu A_n < \infty$
        \end{enumerate}
    \end{definition}

    \begin{definition}
        Пополнение неотрицательной $\sigma$ конечной меры $\mu$ заданной на $\sigma$-алгебре $\text{\textscripta}$ -- это новая мера $\bar{\mu}: \bar{\text{\textscripta}} \rightarrow [0, +\infty]$ такая, что 
        $$\bar{\text{\textscripta}} = \left\{ \bigcup \limits_{k = 1}^{\infty} A_k \,\big|\, \forall k \, \mu^* A_k = 0\right\} \bigcup \text{\textscripta} \equiv \left\{ B \subset \left(\bigcup \text{\textscripta}\right) \,\big|\, \exists A \in \text{\textscripta}\,\, \mu^*(A \Delta B) = 0 \right\}.$$
    \end{definition}
    И если $A$ и $B$ как только что описано, то $\bar{\mu} B = \mu A$.

    Часто вводится $d$-мерная борелевская мера Лебега $\lambda^d_{\mathcal{B}} = \lambda^d \big|_{\mathcal{B}(\mathbb{R}^d)}$.

\end{enumerate}

\begin{definition}
    Пусть $L$ -- линейное (векторное) пространство над $\mathbb{K} \in \{\mathbb{R}, \mathbb{C}\}$, $\ell:L \rightarrow K$ называется линейной (однородной) $\Leftrightarrow \forall v, w \in L, \, \forall c \in \mathbb{K}$, 
    $$\ell(v + cw) = \ell(v) + c\ell(w),$$ 
    $c = 1$ -- аддитивность, $v = 0$ -- однородность 1-ой степени.
\end{definition}

\begin{definition}
    Пусть $L$ как выше, $L_0 \subset L$, $f: L_0  \rightarrow \mathbb{K}$ называется линейной (однородной), если $\exists$ линейная $\ell: L \rightarrow \mathbb{K}$ такая, что $\ell \big|_{L_0} = f$.
\end{definition}

\begin{problem}
    Пусть $K$ -- теоретико-множественное кольцо, $\mu: K \rightarrow \mathbb{K}$ аддитивна, $E \equiv \bigcup K$, $L_0 = \{ \mathds{1}_{A \subset E} \,\big|_:\, A \in K\}, L_0 \subset \mathbb{K}^E$, $\ell_{\mu} (\mathds{1}_{A \subset E} ) = \mu (A) \,\,\forall A \in K$. $\ell_{\mu}$ -- линейная.
\end{problem}

\begin{definition}
    Все линейные и непрерывные продолжения функционала $\ell_{\mu}$ называются интегралами по $\mu$.
\end{definition}

\begin{problem}
    На линейную оболочку $<\mathds{1}_{A \subset E}>_{\mathbb{K}}$ продолжение $\ell_{\mu}$ c сохранением свойства линейности едиственно и элементы $<L_0>_{\mathbb{K}}$ называются простыми интегрируемыми по $\mu$ функции на $E$.
\end{problem}


Если $\mu$ неотрицательная $\sigma$-аддитивная и $\sigma$-конечная на $\sigma$-алгебре $\text{\textscripta}$, то 

$$\text{\textscripta}_{\text{fin,$\mu$}} = \left\{ A \in \text{\textscripta} \,\big|_:\, \mu A < + \infty \right\}$$

(т.е. $A$ -- множество конечной меры $\mu$) является $\delta$-кольцом и простыми интегрируемыми по $\mu$ функциями называются элементы $<\left\{ \mathds{1}_{A \subset E} \,\big|_:\, A \in \text{\textscripta}_{\text{fin,$\mu$}}  \right\}>_{\mathbb{R}} = L_{1, 0} (\mu)$

$$\forall f = \sum \limits_{k = 1}^{n} c_K \mathds{1}_{A_k \subset E} \in L_{1, 0}(\mu)$$ 
такие, что $A_k \in \text{\textscripta}_{\text{fin,$\mu$}}$, $E \equiv \bigcup \text{\textscripta} $

$$ \ell_{\mu}(f) = \sum \limits_{k = 1}^{n} \mu(A_k) \equiv \int \limits_{E} f(x) \mu(dx)$$  

\end{document}
